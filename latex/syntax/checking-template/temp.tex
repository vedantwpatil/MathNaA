\documentclass[12pt,a4paper]{article}

% Essential Math Packages
\usepackage{amsmath,amssymb,amsfonts,amsthm}
\usepackage{mathtools}
\usepackage{thmtools}
\usepackage{tikz}
\usepackage{pgfplots}

% Other Useful Packages
\usepackage[utf8]{inputenc}
\usepackage[T1]{fontenc}
\usepackage[margin=1in]{geometry}
\usepackage{hyperref}
\usepackage{cleveref}

% Math Environments
\newtheorem{theorem}{Theorem}[section]
\newtheorem{lemma}[theorem]{Lemma}
\newtheorem{proposition}[theorem]{Proposition}
\newtheorem{corollary}[theorem]{Corollary}
\newtheorem{definition}[theorem]{Definition}
\newtheorem{example}[theorem]{Example}

% Custom Math Commands
\newcommand{\R}{\mathbb{R}}
\newcommand{\N}{\mathbb{N}}
\newcommand{\Z}{\mathbb{Z}}
\newcommand{\Q}{\mathbb{Q}}
\newcommand{\C}{\mathbb{C}}
\newcommand{\set}[1]{\{#1\}}
\newcommand{\abs}[1]{\left|#1\right|}
\newcommand{\norm}[1]{\left\|#1\right\|}
\newcommand{\inner}[2]{\langle#1,#2\rangle}

\title{Lecture notes}
\author{Vedant Patil}
\date{\today}

\begin{document}

\maketitle

\begin{abstract}
  Covers section 1.3 on the textbook for this date, we covered topics such as summations and sequences 
\end{abstract}

\tableofcontents

\section{First Section}
Your math content goes here.

% Example of a theorem environment
\begin{theorem}
  State your theorem here.
\end{theorem}

\begin{proof}
  Proof goes here.
\end{proof}

% Example of an equation
\begin{equation}
  y = mx + b
\end{equation}

\bibliographystyle{plain}
\bibliography{references}

\end{document}
