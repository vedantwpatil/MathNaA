\documentclass[12pt,a4paper]{article}

% Packages
\usepackage{amsmath,amssymb,amsfonts}
\usepackage{graphicx}
\usepackage[margin=1in]{geometry}
\usepackage{hyperref}
\usepackage{enumitem}
\setlength{\headheight}{15pt}
\usepackage{tcolorbox}
\usepackage{fancyhdr}


% Header and footer
\pagestyle{fancy}
\fancyhf{}
\rhead{Mathematics for Computer Science}
\lhead{Vedant Patil}
\cfoot{\thepage}

% Title
\title{Lecture One - Introduction}
\author{Vedant Patil}
\date{\today}

\begin{document}

\maketitle

\section{Overview}
\begin{tcolorbox}[colback=yellow!10!white,colframe=yellow!50!black,title=Key Points]
  \begin{itemize}
    \item What is a proof
    \item What are propositions 
    \item What are logical deductions 
    \item What are Axioms
  \end{itemize}
\end{tcolorbox}

\section{Detailed Notes}
\subsection{What is a general proof and how to determine the truth}
A proof is a method for establishing the truth
How to determine something is a truth 
\begin{itemize}
  \item Experimentation and observation
  \item Sampling and counter examples 
  \item Judge and juries can also determine something to be the truth or not 
  \item God and religion are something which people lean on to determine the truth 
  \item Somebody with authority is able to determine if something is true or not 
  \item Somebody saying something with conviction can also make people believe it is the truth or not 
\end{itemize}

\subsection{Mathematical Proof}
A mathematical proof is verification of a proposition by chain of logical deductions from a set of axioms. 
A proposition is a statement which is either true or false 
\begin{equation}
  \forall n \in \mathbb{N}, n^2+n+41 \text{ is a prime number.}
\end{equation}

\vspace{0.5cm}

\begin{tabular}{c|c|c}
n & $n^2+n+41$ & Prime \\
\hline
0 & 41 & \( \checkmark \) \\
1 & 43 & \( \checkmark \)\\ 
2 & 47 & \( \checkmark \) \\
3 & 53 & \( \checkmark \)\\
5 & 41 & $\checkmark$ \\
7 & 43 & $\checkmark$ \\
11 & 47 & $\checkmark$ \\
13 & 53 & $\checkmark$ \\
\vdots & \vdots & \vdots \\
29 & 461 & $\checkmark$ \\
\vdots & \vdots & \vdots \\
37 & 1601 & $\checkmark$ \\
40 & 1681 & \( \checkmark \) \\ 
  
\end{tabular}

\medskip
Euluer's Theorem 

\begin{equation}
a^4+b^4+c^4 = d^4
\end{equation}
This was thought to be a valid proposition until it was disproved with the values 
a = 95,500
b = 217,00
c = 414,560
d = 422,481

This is some mathematical notation explaining what we just calculated.
\( \exists a,b,c,d \in \mathbb{N}^+, a^4+b^4+c^4=d^4\)
This means there exists an a,b,c,d in the real integer set that solves the predicate we determined earlier.

\begin{equation}
  313 (x^3+y^3) = z^3 \text{ has no positive integer solutions}
\end{equation}
The earlier example was something which could be possible to brute force using a computer however this second example has too many digits to be able to brute force 
This is a demonstration of something called the elliptic curve, this is important as this is something that is used for encryption and cryptography. This is central to the understanding of factoring large integers 

The main conclusions to make from this is that you cannot just try a few cases and assume that the proof now works for all cases.
\subsection{Implies and Implications}
\begin{equation}
  \forall n \in \mathbb{Z}, n \geq z \Rightarrow n^2 \geq 4 
\end{equation}
Truth Table
\begin{tabular}{c|c|c|c|c}
  p & q & p \( \Rightarrow \) q & q \( \Rightarrow \) p & p \( \Leftrightarrow \) q\\
  \hline
  T & T & T & T & T\\ 
  T & F & F & T & F\\ 
  F & T & T & F & F \\ 
  F & F & T & T & T 
\end{tabular}

This can be seen with the statement 
\begin{equation}
  \text{Pigs fly } \Rightarrow \text{I am king}
\end{equation}

This is a true statement because pigs don't fly. So because pigs don't fly, it doesn't matter if I am king or not

\begin{equation}
  \forall n \in \mathbb{Z}, n \geq 2 \Leftrightarrow n^2 \geq 4 
\end{equation}
This is false when n = -3 
The \( \leftrightarrow \) means if and only if, this requires us to check both sides of the statement

\section{Important Formulas/Theorems}
\begin{tcolorbox}[colback=blue!5!white,colframe=blue!75!black,title=Key Formula/Theorem]
  \begin{itemize}
    \item A mathematical proof is verification of a proposition by chain of logical deductions from a set of axioms
    \item A proposition is a statement which is either true or false 
    \item Predicate - proposition whose truth depends on the value of variable(s)
    \item elliptic curve something which is used in cryptography
    \item An implication \( p \Rightarrow \) is true if p is False or q is True 

  \end{itemize}
\end{tcolorbox}

\section{Examples}
\begin{tcolorbox}
  Write an example here
\end{tcolorbox}

\section{Questions/Topics for Further Study}
\begin{itemize}
  \item Question or topic for further study
\end{itemize}

\end{document}
