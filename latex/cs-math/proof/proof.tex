\documentclass{article}
\usepackage{amsmath}
\usepackage{amssymb}
\usepackage{amsthm}

\newtheorem{theorem}{Theorem}
\newtheorem{lemma}[theorem]{Lemma}

\begin{document}

\section{A Proof in Propositional Logic}

\begin{theorem}
If $p \implies q$ and $q \implies r$, then $p \implies r$.
\end{theorem}

\begin{proof}
We will prove this theorem using the following steps:

\begin{enumerate}
    \item $p \implies q$ (Given)
    \item $q \implies r$ (Given)
    \item $p$ (Assumption)
    \item $q$ (Modus Ponens, 1 and 3)
    \item $r$ (Modus Ponens, 2 and 4)
\end{enumerate}

From steps 3-5, we have shown that assuming $p$ leads to $r$. Therefore, by the definition of implication:

\[
p \implies r
\]

This completes the proof.
\end{proof}

\begin{lemma}
$\neg(p \land q) \iff (\neg p \lor \neg q)$
\end{lemma}

\begin{proof}
We will prove this equivalence using a truth table:

\begin{center}
\begin{tabular}{|c|c|c|c|c|c|}
\hline
$p$ & $q$ & $p \land q$ & $\neg(p \land q)$ & $\neg p$ & $\neg q$ & $\neg p \lor \neg q$ \\
\hline
T & T & T & F & F & F & F \\
T & F & F & T & F & T & T \\
F & T & F & T & T & F & T \\
F & F & F & T & T & T & T \\
\hline
\end{tabular}
\end{center}

As we can see, the truth values for $\neg(p \land q)$ and $(\neg p \lor \neg q)$ are identical for all possible combinations of $p$ and $q$. Therefore, these expressions are logically equivalent.
\end{proof}

\begin{theorem}
If $p \implies (q \implies r)$, then $(p \land q) \implies r$.
\end{theorem}

\begin{proof}
We will prove this using natural deduction:

\begin{enumerate}
    \item $p \implies (q \implies r)$ (Given)
    \item $p \land q$ (Assumption)
    \item $p$ (Conjunction Elimination, 2)
    \item $q \implies r$ (Modus Ponens, 1 and 3)
    \item $q$ (Conjunction Elimination, 2)
    \item $r$ (Modus Ponens, 4 and 5)
\end{enumerate}

From steps 2-6, we have shown that assuming $p \land q$ leads to $r$. Therefore, by the definition of implication:

\[
(p \land q) \implies r
\]

This completes the proof.
\end{proof}

\end{document}
