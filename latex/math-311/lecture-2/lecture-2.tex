\documentclass[12pt,a4paper]{article}

% Packages
\usepackage{amsmath,amssymb,amsfonts}
\usepackage{graphicx}
\usepackage[margin=1in]{geometry}
\usepackage{hyperref}
\usepackage{enumitem}
\usepackage{tcolorbox}
\usepackage{fancyhdr}
\setlength{\headheight}{15pt}


% Header and footer
\pagestyle{fancy}
\fancyhf{}
\rhead{MATH 311}
\lhead{Vedant Patil}
\cfoot{\thepage}

% Title
\title{Lecture Notes: Section 2.4}
\author{Vedant Patil}
\date{\today}

\begin{document}

\maketitle

\section{Overview}
\begin{tcolorbox}[colback=yellow!10!white,colframe=yellow!50!black,title=Key Points]
  \begin{itemize}
    \item Key point 3
  \end{itemize}
\end{tcolorbox}

\section{Detailed Notes}
\subsection{Subtopic 1}
When an experiament is performed there are a number of possible outcomes which may occur, we call those events.  \\ 
For example say you toss a coin twice, these are some of the possible events 
\begin{itemize}
  \item A - Get the same on both flips 
  \item B - Different on the 2 flips 
  \item C - First flip is H 
  \item D - First flip is T 
  \item F - Second flip is H 
  \item G - Second flip is T 
  \item H - Don't get TH 
\end{itemize}
There is a difference between those events and these, we denote them with the letter E for special cases  \\
\begin{itemize}
  \item \( E_{1} \)- HH 
  \item \( E_{2} \)- HT 
  \item \( E_{3} \)- TH 
  \item \( E_{4} \) - TT 
\end{itemize}

The sample space associated with a experiament is associated with all possible sample points \\ 

The sample space will be Denoate by \( \mathbb{S} \) similar to the universal set 

\begin{equation}
  S = \{HH, HT, TH, TT\} 
\end{equation}

An event in discrete sample space \( \mathbb{S} \) is a collection of sample poitns that is a subset of \( \mathbb{S} \)

For the events described previously, we have 

\begin{equation}
\begin{aligned}
A &= \{HH, TT\} \\
B &= \{HT, TH\} \\
C &= \{HH, HT\} \\
D &= \{TH, TT\} \\
F &= \{HH, TH\} \\
G &= \{HT, TT\} \\
H &= \{HH, HT, TT\} \\
E_{1} &= \{HH\} \\
E_{2} &= \{HT\} \\
E_{3} &= \{TH\} \\
E_{4} &= \{TT\}
\end{aligned}
\end{equation}

Now we will talk about probablity associated with events 

\subsection{Subtopic 2}
Your notes for subtopic 2

\section{Important Formulas/Theorems/Definitions}
\begin{tcolorbox}[colback=blue!5!white,colframe=blue!75!black,title=Key Formula/Theorem]
  \begin{itemize}
    \item The discrete case is a probalistic model for an experiament
    \item Definition: An experiament is the proccess by which an observation is made 
      \begin{itemize}
        \item Tossing a coin or dye a number of times 
        \item Measuring the height of a group of people 
        \item Counting the number of bacteria in a certain sample 
      \end{itemize}
    \item Definition: A simple event is a event which cannot be decomposed. Each simple event corresponds to one and only one sample point. The letter E wwith a subscript is used to denote a single simple event/sample point.
    \item Definition: A discrete sample space is ore that contains either a finite or a countable number of distinct sample points 
    \item Suppose \( \mathbb{S} \) is a sample space associated with a expierament. To every event, \( A \) in \( \mathbb{S} \) (for every \( A \subsetneqq S \)) We assign a number, P(A) the probablity of A, so the following Axioms hold 
      \begin{itemize}
        \item Axiom 1: \( P(A) \geq 0 \text{ for all } A \subsetneqq S \)
        \item Axiom 2: \( P(S) = 1 \)
        \item Axiom 3: If \( A_{1}, A_{2} \text{are pairwise mutually exclusive } \)
      \end{itemize}
  \end{itemize}
\end{tcolorbox}

\section{Questions/Topics for Further Study}
\begin{itemize}
  \item Question or topic for further study
\end{itemize}

\end{document}
