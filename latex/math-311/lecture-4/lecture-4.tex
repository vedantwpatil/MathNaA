\documentclass[12pt,a4paper]{article}

% Packages
\usepackage{amsmath,amssymb,amsfonts}
\usepackage{graphicx}
\usepackage[margin=1in]{geometry}
\usepackage{hyperref}
\usepackage{enumitem}
\usepackage{tcolorbox}
\usepackage{fancyhdr}
\setlength{\headheight}{15pt}


% Header and footer
\pagestyle{fancy}
\fancyhf{}
\rhead{MATH 311}
\lhead{Vedant Patil}
\cfoot{\thepage}

% Title
\title{Lecture Notes: Section 2.6: Tools for Counting Sample Points}
\author{Vedant Patil}
\date{\today}

\begin{document}

\maketitle

\section{Overview}
\begin{tcolorbox}[colback=yellow!10!white,colframe=yellow!50!black,title=Key Points]
  \begin{itemize}
    \item Being able to calculate the amount of possibilities given specific information about things like does the order matter and other relevant selection information 
  \end{itemize}
\end{tcolorbox}

\section{Detailed Notes}
\subsection{Theorem 2.1}
This section will just hold a number of examples demonstrating the theorem 
\vspace{12pt}
Shapes and colors \\ 
Square, Circle and triangle in red, yellow, blue, purple with one object of every color there are 12 possibilities because \( 3 * 4 \)

\vspace{12pt} 
Toss 2 6 sided dice, how many possible outcomes are there. \\ 
\begin{itemize}
  \item 6 for the first die 
  \item 6 for the second die 
  \item 36 total possible outcomes 
\end{itemize}

\vspace{12pt}
List the birthdays for a group of 20 people. Assume no leap years 
\begin{itemize}
  \item \( 365^{20} \) possibilities
\end{itemize}

Assume all birthdays are equally likely, what is the probability that all 20 people have different birthdays \\
\vspace{12pt}

\( 365 * 364 * 363\ldots 346 \) this is how many sample points there are in which everyone has their own unique birthday 
\begin{itemize}
  \item \( \implies  \) The probability is \( \frac{365\ldots 346}{365^{20}} \)
  \item \( = \frac{365}{365} * \frac{364}{365} \ldots \frac{346}{365} \)
\end{itemize}

Definition: An ordered arrangement of \( r \) distinct objects is called a permutation. The number of ways of ordering \( n \) distinct objects, taking r at a time is \( P^{n}_{r} \)

Example: The names of 3 employes of a company of 30 will be drawn at random sequntially. The first picked gets 100 dollars, second gets 50 and the third gets 25 
\begin{equation}
  P^{30}_{3} = 30 * 29 * 28  = \frac{30!}{27!}
\end{equation}

\section{Important Formulas/Theorems/Definitions}
\begin{tcolorbox}[colback=blue!5!white,colframe=blue!75!black,title=Key Formula/Theorem]
  \begin{itemize}
    \item Theorem 2.1: With M elements \( a_{1},a_{2},a_{3} \ldots a_{m} \) and \( n  \) elements \( b_{1},b_{2},b_{3} \ldots b_{n} \) it is possible to form \( mn  \) pairs with one element from each group 
    \item Definition: An ordered arrangement of \( r \) distinct objects is called a permutation. The number of ways of ordering \( n \) distinct objects, taking r at a time is \( P^{n}_{r} \)
  \end{itemize}
\end{tcolorbox}

\section{Examples}
\begin{tcolorbox}
  Write an example here
\end{tcolorbox}

\section{Questions/Topics for Further Study}
\begin{itemize}
  \item Question or topic for further study
\end{itemize}

\end{document}
