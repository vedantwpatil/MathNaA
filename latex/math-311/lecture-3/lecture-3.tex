\documentclass[12pt,a4paper]{article}

% Packages
\usepackage{amsmath,amssymb,amsfonts}
\usepackage{graphicx}
\usepackage[margin=1in]{geometry}
\usepackage{hyperref}
\usepackage{enumitem}
\usepackage{tcolorbox}
\usepackage{fancyhdr}
\setlength{\headheight}{15pt}


% Header and footer
\pagestyle{fancy}
\fancyhf{}
\rhead{MATH 311}
\lhead{Vedant Patil}
\cfoot{\thepage}

% Title
\title{Lecture Notes: Section 2.6}
\author{Vedant Patil}
\date{\today}

\begin{document}

\maketitle

\section{Overview}
\begin{tcolorbox}[colback=yellow!10!white,colframe=yellow!50!black,title=Key Points]
  \begin{itemize}
    \item Key point 3
  \end{itemize}
\end{tcolorbox}

\section{Detailed Notes}
\subsection{Subtopic 1}
Your notes for subtopic 1

\subsection{Subtopic 2}
Your notes for subtopic 2

\section{Important Formulas/Theorems/Definitions}
\begin{tcolorbox}[colback=blue!5!white,colframe=blue!75!black,title=Key Formula/Theorem]
  \begin{theorem}[2.1]
    With M elements \( a_{1}, a_{2}, \ldots a_{m} \) and n elements \( b_{1}, b_{2}, \ldots b_{n} \) it is possible to form mn pairs with one elemnt from each group 
  \end{theorem}
\end{tcolorbox}

\section{Examples}
\begin{tcolorbox}
  \begin{itemize}
    \item Shapes and colors, Square, Circle and triangle in red, yellow, blue, purple with one objkect of every color there are 12 possibilities where m = shapes and n = colors being 3 * 4 = 12 possibilities 
    \item Toss 2 6 sided dice, how many possible outcomes are there. 
      \begin{itemize}
        \item There are 6 for the first die 
        \item There are 6 for the second die 
        \item This comes out to 36 total opportunities 
      \end{itemize}
  \end{itemize}
\end{tcolorbox}

\section{Questions/Topics for Further Study}
\begin{itemize}
  \item Question or topic for further study
\end{itemize}

\end{document}
