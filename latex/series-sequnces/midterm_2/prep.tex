\documentclass{article}
\usepackage{amsmath}
\usepackage{amssymb} % For additional math symbols
\usepackage{hyperref} % For hyperlinks
\usepackage{geometry} % For better page layout
\geometry{margin=1in}

\begin{document}

\title{Midterm II Preparation Notes}
\author{Vedant Patil}
\date{\today}
\maketitle

\section*{Notes on Convergence Tests}

\subsection*{Divergence Test}
The divergence test (also known as the nth-term test for divergence) can be used to check if a series is diverging. With the divergence test, you can only conclude that the series is divergent if the test fails.

If the limit of the nth term of a series \(\sum a_n\) does not approach zero, i.e.,
\[
\lim_{n \to \infty} a_n \neq 0,
\]
then the series \(\sum a_n\) diverges.

\textbf{Example:}
Consider the series \(\sum \frac{1}{n}\). Here, \(a_n = \frac{1}{n}\). We have:
\[
\lim_{n \to \infty} \frac{1}{n} = 0.
\]
Since the limit is zero, the divergence test is inconclusive. However, for the series \(\sum 1\), where \(a_n = 1\), we have:
\[
\lim_{n \to \infty} 1 = 1 \neq 0,
\]
so the series diverges by the divergence test.

\subsection*{Comparison Test}
The Comparison Test can be used to determine if a series is converging or diverging by comparing it to another series which we are able to evaluate 

If \(0 \leq a_n \leq b_n\) for all \(n\) and \(\sum b_n\) converges, then \(\sum a_n\) also converges.

Conversely, if \(\sum b_n\) diverges and \(a_n \geq b_n \geq 0\), then \(\sum a_n\) also diverges.

\textbf{Example:}
Consider the series \(\sum \frac{1}{n^2}\) and \(\sum \frac{1}{n^3}\). Since \(\frac{1}{n^3} \leq \frac{1}{n^2}\) for all \(n \geq 1\) and \(\sum \frac{1}{n^2}\) converges (p-series with \(p > 1\)), by the Comparison Test, \(\sum \frac{1}{n^3}\) also converges.

\subsection*{Ratio Test}
The Ratio Test involves the limit of the absolute value of the ratio of consecutive terms in a series. For a series \(\sum a_n\), if \(\lim_{n \to \infty} \left| \frac{a_{n+1}}{a_n} \right| = L\):
\begin{itemize}
    \item If \(L < 1\), the series converges absolutely.
    \item If \(L > 1\) or \(L = \infty\), the series diverges.
    \item If \(L = 1\), the test is inconclusive.
\end{itemize}

\textbf{Example:}
Consider the series \(\sum \frac{n!}{n^n}\). Let \(a_n = \frac{n!}{n^n}\). Then,
\[
\left| \frac{a_{n+1}}{a_n} \right| = \left| \frac{(n+1)!}{(n+1)^{n+1}} \cdot \frac{n^n}{n!} \right| = \left| \frac{(n+1) \cdot n!}{(n+1)^{n+1}} \cdot \frac{n^n}{n!} \right| = \left| \frac{n^n}{(n+1)^n} \cdot \frac{1}{n+1} \right| \approx \frac{1}{e} < 1
\]
Thus, the series converges by the Ratio Test.

\subsection*{Root Test}
The Root Test uses the \(n\)-th root of the absolute value of the terms in a series. For a series \(\sum a_n\), if \(\lim_{n \to \infty} \sqrt[n]{|a_n|} = L\):
\begin{itemize}
    \item If \(L < 1\), the series converges absolutely.
    \item If \(L > 1\) or \(L = \infty\), the series diverges.
    \item If \(L = 1\), the test is inconclusive.
\end{itemize}

\textbf{Example:}
Consider the series \(\sum \left( \frac{1}{2} \right)^n\). Let \(a_n = \left( \frac{1}{2} \right)^n\). Then,
\[ \sqrt[n]{|a_n|} = \sqrt[n]{\left( \frac{1}{2} \right)^n} = \frac{1}{2} \]
Since \(\frac{1}{2} < 1\), the series converges by the Root Test.

\subsection*{Alternating Series Test}
The Alternating Series Test applies to series of the form \(\sum (-1)^n a_n\) or \(\sum (-1)^{n+1} a_n\) where \(a_n \geq 0\). The series converges if:
\begin{itemize}
    \item \(a_n\) is monotonically decreasing, and
    \item \(\lim_{n \to \infty} a_n = 0\).
\end{itemize}

\textbf{Example:}
Consider the series \(\sum (-1)^n \frac{1}{n}\). Here, \(a_n = \frac{1}{n}\), which is monotonically decreasing and \(\lim_{n \to \infty} \frac{1}{n} = 0\). Thus, the series converges by the Alternating Series Test.

\subsection*{Absolute Convergence}
A series \(\sum a_n\) converges absolutely if the series of absolute values \(\sum |a_n|\) converges. Absolute convergence implies convergence, but not vice versa.

\textbf{Example:}
Consider the series \(\sum (-1)^n \frac{1}{n^2}\). The series of absolute values is \(\sum \frac{1}{n^2}\), which converges (p-series with \(p > 1\)). Therefore, \(\sum (-1)^n \frac{1}{n^2}\) converges absolutely.

\section*{Notes on Maclaurin and Taylor Polynomials}

\subsection*{Maclaurin Polynomials}
A Maclaurin polynomial is a special case of the Taylor polynomial centered at \(x = 0\). The \(n\)-th degree Maclaurin polynomial for a function \(f(x)\) is:
\[ P_n(x) = f(0) + f'(0)x + \frac{f''(0)}{2!}x^2 + \cdots + \frac{f^{(n)}(0)}{n!}x^n \]

\textbf{Example:}
For \(f(x) = e^x\), the Maclaurin polynomial of degree 3 is:
\[ P_3(x) = 1 + x + \frac{x^2}{2!} + \frac{x^3}{3!} = 1 + x + \frac{x^2}{2} + \frac{x^3}{6} \]

\subsection*{Taylor Polynomials}
A Taylor polynomial approximates a function \(f(x)\) near a point \(a\). The \(n\)-th degree Taylor polynomial for \(f(x)\) centered at \(a\) is:
\[ P_n(x) = f(a) + f'(a)(x-a) + \frac{f''(a)}{2!}(x-a)^2 + \cdots + \frac{f^{(n)}(a)}{n!}(x-a)^n \]

\textbf{Example:}
For \(f(x) = \sin(x)\) centered at \(a = \pi/4\), the Taylor polynomial of degree 2 is:
\[ P_2(x) = \sin\left(\frac{\pi}{4}\right) + \cos\left(\frac{\pi}{4}\right)(x - \frac{\pi}{4}) - \frac{\sin\left(\frac{\pi}{4}\right)}{2!}(x - \frac{\pi}{4})^2 \]

\subsection*{Estimating Function Values}
To estimate the value of a function \(f(x)\) to within 3 decimal places using either a Taylor or Maclaurin polynomial, follow these steps:
\begin{enumerate}
    \item Choose the appropriate polynomial (Taylor or Maclaurin) based on the function and the point of interest.
    \item Determine the degree \(n\) of the polynomial such that the remainder term \(R_n(x)\) is less than 0.001.
    \item Calculate the polynomial \(P_n(x)\) and use it to approximate \(f(x)\).
\end{enumerate}

\textbf{Example:}
Estimate \(e^{0.1}\) using the Maclaurin polynomial for \(e^x\).

The Maclaurin series for \(e^x\) is:
\[ e^x = \sum_{n=0}^{\infty} \frac{x^n}{n!} \]

To estimate \(e^{0.1}\) to within 3 decimal places, we need to find the degree \(n\) such that the remainder term \(R_n(0.1)\) is less than 0.001. For \(e^x\), the remainder term is:
\[ R_n(x) = \frac{e^c x^{n+1}}{(n+1)!} \]
where \(c\) is some value between 0 and 0.1.

Using the first four terms of the series:
\[ P_3(0.1) = 1 + 0.1 + \frac{0.1^2}{2!} + \frac{0.1^3}{3!} = 1 + 0.1 + 0.005 + 0.0001667 = 1.1051667 \]

The actual value of \(e^{0.1}\) is approximately 1.1051709, so the error is less than 0.001, and the estimate is accurate to within 3 decimal places.

\section*{Notes on Maclaurin and Taylor Series}

\subsection*{Maclaurin Series}
The Maclaurin series is the infinite series representation of a function \(f(x)\) centered at \(x = 0\):
\[ f(x) = \sum_{n=0}^{\infty} \frac{f^{(n)}(0)}{n!} x^n \]

\textbf{Example:}
For \(f(x) = \cos(x)\), the Maclaurin series is:
\[ \cos(x) = \sum_{n=0}^{\infty} \frac{(-1)^n}{(2n)!} x^{2n} \]

\subsection*{Taylor Series}
The Taylor series is the infinite series representation of a function \(f(x)\) centered at \(a\):
\[ f(x) = \sum_{n=0}^{\infty} \frac{f^{(n)}(a)}{n!} (x-a)^n \]

\textbf{Example:}
For \(f(x) = e^x\) centered at \(a = 1\), the Taylor series is:
\[ e^x = \sum_{n=0}^{\infty} \frac{e}{n!} (x-1)^n \]

\section*{Notes on Power Series}
A power series is an infinite series of the form:
\[ \sum_{n=0}^{\infty} c_n (x-a)^n \]
where \(c_n\) are coefficients and \(a\) is the center of the series. The series converges within a certain radius \(R\) around \(a\), known as the radius of convergence.

\textbf{Example:}
Consider the power series \(\sum_{n=0}^{\infty} \frac{x^n}{n!}\). This series converges for all \(x\) (radius of convergence \(R = \infty\)).

\subsection*{Properties of Power Series}
1. *Term-by-Term Differentiation and Integration*: If a power series converges within its radius of convergence, it can be differentiated and integrated term-by-term within that interval.
   \[ \frac{d}{dx} \left( \sum_{n=0}^{\infty} c_n (x-a)^n \right) = \sum_{n=1}^{\infty} n c_n (x-a)^{n-1} \]
   \[ \int \left( \sum_{n=0}^{\infty} c_n (x-a)^n \right) dx = C + \sum_{n=0}^{\infty} \frac{c_n (x-a)^{n+1}}{n+1} \]

2. **Uniform Convergence**: A power series converges uniformly on any closed interval within its radius of convergence.

\subsection*{Examples of Power Series}
1. **Geometric Series**: 
   \[ \sum_{n=0}^{\infty} x^n = \frac{1}{1-x} \quad \text{for} \quad |x| < 1 \]

2. **Exponential Function**:
   \[ e^x = \sum_{n=0}^{\infty} \frac{x^n}{n!} \]

3. **Sine and Cosine Functions**:
   \[ \sin(x) = \sum_{n=0}^{\infty} (-1)^n \frac{x^{2n+1}}{(2n+1)!} \]
   \[ \cos(x) = \sum_{n=0}^{\infty} (-1)^n \frac{x^{2n}}{(2n)!} \]

\section*{Notes on Convergence of Taylor Series}
The convergence of a Taylor series depends on the function being represented and the point around which the series is centered. For a function \(f(x)\) with a Taylor series centered at \(a\):
\[ f(x) = \sum_{n=0}^{\infty} \frac{f^{(n)}(a)}{n!} (x-a)^n \]

The series converges within a certain interval around \(a\), known as the interval of convergence. The radius of convergence \(R\) is the distance from \(a\) to the boundary of this interval. The series converges absolutely for \(|x - a| < R\) and diverges for \(|x - a| > R\).

To determine the radius of convergence, one can use the Ratio Test or the Root Test. For the Ratio Test, consider the limit:
\[ \lim_{n \to \infty} \left| \frac{a_{n+1}}{a_n} \right| = L \]
where \(a_n = \frac{f^{(n)}(a)}{n!} (x-a)^n\). The radius of convergence \(R\) is given by:
\[ R = \frac{1}{L} \]

For the Root Test, consider the limit:
\[ \lim_{n \to \infty} \sqrt[n]{|a_n|} = L \]
where \(a_n = \frac{f^{(n)}(a)}{n!} (x-a)^n\). The radius of convergence \(R\) is given by:
\[ R = \frac{1}{L} \]

Within the interval of convergence, the Taylor series converges to the function \(f(x)\). At the endpoints of the interval, convergence must be checked separately.

\textbf{Example:}
For the Taylor series of \(f(x) = \ln(1+x)\) centered at \(a = 0\):
\[ \ln(1+x) = \sum_{n=1}^{\infty} (-1)^{n+1} \frac{x^n}{n} \]
The radius of convergence is \(R = 1\), so the series converges for \(|x| < 1\).

\end{document}
