\documentclass[12pt,a4paper]{article}

% Packages
\usepackage{amsmath,amssymb,amsfonts}
\usepackage{graphicx}
\usepackage[margin=1in]{geometry}
\usepackage{hyperref}
\usepackage{enumitem}
\usepackage{tcolorbox}
\usepackage{fancyhdr}
\setlength{\headheight}{15pt}


% Header and footer
\pagestyle{fancy}
\fancyhf{}
\rhead{MATH 220}
\lhead{Vedant Patil}
\cfoot{\thepage}

% Title
\title{Lecture Notes: Logic}
\author{Vedant Patil}
\date{\today}

\begin{document}

\maketitle

\section{Overview}
\begin{tcolorbox}[colback=yellow!10!white,colframe=yellow!50!black,title=Key Points]
  \begin{itemize}
    \item Understanding and combining logic statements with other logical expressions 
  \end{itemize}
\end{tcolorbox}

\section{Detailed Notes}
\subsection{Logical Statements}
A statement is a declarative statement assertion that is either true or false but not both (T, F \( \implies 1,0 \)) are called boolean values \\ 

Example 

\begin{itemize}
  \item Number 4 is even (T) 
  \item Number 7 is odd (T) 
  \item number 5 is divisible by 3 (F) 
\end{itemize}

An open sentence is a statement that contains one or more verbs \\ 

The range of possible value of these variables is called the domain of open sentence, often denoted by \( \mathbb{S} \)

Number \( \mathbb{N} \) is even so we usually denote open sentences by \( p(x) : x \in \mathbb{S} \)

\subsection{Combining Statements}
\subsubsection{Negation}
Negation: given a statement \( P \) it's negation "Not \( P \)" is denoted by \( \neg P \) \\ 
Example 
\begin{itemize}
  \item Number 4 is not even 
  \item Number 4 is odd 
\end{itemize}

This is a more concrete example 
\vspace{8pt}
Establishing knowledge 
\begin{itemize}
  \item \( P(N) \): Number n is odd, \( S = \{1,2,3,4\}   \)
  \item \( \neg P(N) \): number n is even, \( S = \{1,2,3,4\}   \)
\end{itemize}

\begin{itemize}
  \item  \( \neg P(1): F \)
  \item  \( \neg P(2): T \)
  \item  \( \neg P(3): F \)
  \item  \( \neg P(4): T \)
\end{itemize}

\subsubsection{Disjunction}
Given statements P, Q their disjunction, denoted by \( P \lor Q \) is a statement P or Q

Example)
\vspace{12pt}
\begin{itemize}
  \item P: I am going to be at a bar tonight 
  \item Q: I am going to the movies tonight 
\end{itemize}

\vspace{12pt}

Truth Table 

\begin{tabular}{|c|c|}
\hline
p & q \\
\hline
T & T \\
T & F \\
F & T \\
F & F \\
\hline
\end{tabular}

\subsubsection{Conjucntion}
Given statements \( P,Q \) their conjunction denoted by \( P \land Q\) is the statement P and Q and is true only when the both of them are true 

\section{Important Formulas/Theorems/Definitions}
\begin{tcolorbox}[colback=blue!5!white,colframe=blue!75!black,title=Key Formula/Theorem]
  State an important formula or theorem here
\end{tcolorbox}

\section{Examples}
\begin{tcolorbox}
  Example 1: 
  \begin{equation}
    P(N): \text{number n is odd,} s = \{1,2,3,4\}  
  \end{equation}
  \begin{itemize}
    \item \( P(1): T \)
    \item \( P(2): F \)
    \item \( P(3): T \)
    \item \( P(4): F \)
  \end{itemize}

  Example 2: 
  \begin{equation}
    P(x,y) \text{ is divisible by } x + y
  \end{equation}
  \begin{equation}
    S = \{a,b\}  a \in \{1,2,3\}, b \in \{1,2\}  \\
  \end{equation}
  \begin{itemize}
    \item P(1,1) = 1 * 1 \text{is } -1 -1 -1 + 1 \text{ this statement is false } 
    \item P(1,2) = 1 * 2 \text{is } -1 -1 -1 + 1 \text{ this statement is false } 
    \item P(2,1) \text{ this statement is false } 
  \end{itemize}

\end{tcolorbox}

\section{Questions/Topics for Further Study}
\begin{itemize}
  \item Question or topic for further study
\end{itemize}

\end{document}
