\documentclass[12pt,a4paper]{article}

% Packages
\usepackage{amsmath,amssymb,amsfonts}
\usepackage{graphicx}
\usepackage[margin=1in]{geometry}
\usepackage{hyperref}
\usepackage{enumitem}
\usepackage{tcolorbox}
\usepackage{fancyhdr}
\setlength{\headheight}{15pt}


% Header and footer
\pagestyle{fancy}
\fancyhf{}
\rhead{Math 220}
\lhead{Vedant Patil}
\cfoot{\thepage}

% Title
\title{Lecture Notes: Set Operations}
\author{Vedant Patil}
\date{\today}

\begin{document}

\maketitle

\section{Overview}
\begin{tcolorbox}[colback=yellow!10!white,colframe=yellow!50!black,title=Key Points]
  \begin{itemize}
    \item Set Operations 
  \end{itemize}
\end{tcolorbox}

\section{Detailed Notes}
\subsection{Set Operations}


\subsection{Subtopic 2}
Your notes for subtopic 2

\section{Important Formulas/Theorems/Definitions}
\begin{tcolorbox}[colback=blue!5!white,colframe=blue!75!black,title=Key Formula/Theorem]
  \begin{itemize}
    \item  Union: If A B are sets then their union \( A \cup B \) is a set defined by 
  \begin{equation}
     A \cup B = \{x: x \in \text{ or } x \in B \}   
  \end{equation}
  \end{itemize}
\end{tcolorbox}

\section{Examples}
\begin{tcolorbox}
  \begin{itemize}
    \item \( A = \{1,2,4,8\} = {x:x=2^i; i=0, 1, 2, 3}  \)
  \end{itemize}
\end{tcolorbox}

\section{Questions/Topics for Further Study}
\begin{itemize}
  \item Question or topic for further study
\end{itemize}

\end{document}
