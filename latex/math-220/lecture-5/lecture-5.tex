\documentclass[12pt,a4paper]{article}

% Packages
\usepackage{amsmath,amssymb,amsfonts}
\usepackage{graphicx}
\usepackage[margin=1in]{geometry}
\usepackage{hyperref}
\usepackage{enumitem}
\usepackage{tcolorbox}
\usepackage{fancyhdr}
\setlength{\headheight}{15pt}

% Header and footer
\pagestyle{fancy}
\fancyhf{}
\rhead{MATH 220}
\lhead{Vedant Patil}
\cfoot{\thepage}

% Title
\title{Lecture Notes: Implies and Biconditional}
\author{Vedant Patil}
\date{\today}

\begin{document}

\maketitle

\section{Overview}
\begin{tcolorbox}[colback=yellow!10!white,colframe=yellow!50!black,title=Key Points]
  \begin{itemize}
    \item Understanding the implies operator and its truth table
    \item Understanding the biconditional operator and how to use it
  \end{itemize}
\end{tcolorbox}

\section{Detailed Notes}
\subsection{Implies}
For propositions \(P\) and \(Q\), the implication is denoted by \(P \implies Q\) and read as "if P then Q".

\begin{center}
\begin{tabular}{|c|c|c|}
\hline
P & Q & P \rightarrow Q \\
\hline
T & T & T \\
T & F & F \\
F & T & T \\
F & F & T \\
\hline
\end{tabular}
\end{center}

The important conclusion is that when P is true, Q cannot be false for the implication to be true.

Example: If 3 is even then \(3 * 15\) is even 
\begin{itemize}
  \item \(P_1 = \text{"3 is even"}\)
  \item \(Q_1 = \text{"3 * 15 is even"}\)
  \item \(P_1 \implies Q_1\) is true 
\end{itemize}

P is false, so we can make any conclusion we want (the implication is vacuously true).

\subsection{Biconditional}
The biconditional operator, denoted by \(\leftrightarrow\) or \(\iff\), is read as "if and only if" (often abbreviated as "iff").

\begin{center}
\begin{tabular}{|c|c|c|}
\hline
P & Q & P \leftrightarrow Q \\
\hline
T & T & T \\
T & F & F \\
F & T & F \\
F & F & T \\
\hline
\end{tabular}
\end{center}

The biconditional is true when both P and Q have the same truth value.

Properties of the biconditional:
\begin{itemize}
  \item Symmetry: \(P \leftrightarrow Q\) is equivalent to \(Q \leftrightarrow P\)
  \item \(P \leftrightarrow Q\) is equivalent to \((P \implies Q) \land (Q \implies P)\)
\end{itemize}

Example: \(x^2 = 4 \iff x = \pm 2\)
This statement is true because:
\begin{itemize}
  \item If \(x^2 = 4\), then \(x = \pm 2\)
  \item If \(x = \pm 2\), then \(x^2 = 4\)
\end{itemize}

\section{Important Formulas/Theorems/Definitions}
\begin{tcolorbox}[colback=blue!5!white,colframe=blue!75!black,title=Key Formula/Theorem]
  \(P \leftrightarrow Q \equiv (P \implies Q) \land (Q \implies P)\)
\end{tcolorbox}

\section{Examples}
\begin{tcolorbox}
Demonstrating Implies and Biconditional
\begin{itemize}
  \item P(x): \(x^2 = 16\)
  \item Q(x): \(|x| = 4\)
  \item \(P(x) \implies Q(x)\) is true for all x
  \item \(P(x) \leftrightarrow Q(x)\) is true for all real x
\end{itemize}
\end{tcolorbox}

\section{Questions/Topics for Further Study}
\begin{itemize}
  \item How does the biconditional relate to logical equivalence?
  \item Explore the use of biconditionals in mathematical definitions and theorems.
\end{itemize}

\end{document}
