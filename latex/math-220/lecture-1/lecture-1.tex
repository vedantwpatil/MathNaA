\documentclass[12pt,a4paper]{article}

% Packages
\usepackage{amsmath,amssymb,amsfonts}
\usepackage{graphicx}
\usepackage[margin=1in]{geometry}
\usepackage{hyperref}
\usepackage{enumitem}
\usepackage{tcolorbox}
\usepackage{fancyhdr}
\setlength{\headheight}{15pt}


% Header and footer
\pagestyle{fancy}
\fancyhf{}
\rhead{Math 220}
\lhead{Vedant Patil}
\cfoot{\thepage}

% Title
\title{Lecture Notes: Introduction to Sets and Set Notation}
\author{Vedant Patil}
\date{\today}

\begin{document}

\maketitle

\section{Overview}
\begin{tcolorbox}[colback=yellow!10!white,colframe=yellow!50!black,title=Key Points]
  \begin{itemize}
    \item Set Rules
    \item Set Notation 
    \item Set Cardinality
  \end{itemize}
\end{tcolorbox}

\section{Detailed Notes}
\subsection{Set Rules}
\begin{itemize}
  \item Set Definition: A collection of objects called elements or members of that set. 
  \item Notation for a set is usually captial letters 
    \begin{itemize}
      \item Example: \( A, B, C \)
    \end{itemize}
    \item Lowercase letters represent elements in the set 
      \begin{itemize}
        \item Example: \( a, b, c \) 
      \end{itemize}
    \item If a is a element of A 
      \begin{itemize}
        \item Example: \( a \in A \)
      \end{itemize}
    \item If a is not an element of A we write 
      \begin{itemize}
        \item Example: \( a \notin A \)
      \end{itemize}
    \item \( A = { x : p(x)} \)
    \item \(  A = {x : |x| 2} = \{-2, 2\}= \{ 2, -2 \} \)
\end{itemize}

\subsection{Equality of Sets and Cordinality of Sets}
\begin{itemize}
  \item \( A = B \) if \( A \) and \( B \) have the same elements 
  \item A set with no elements is called an empty set or a void set and is denoted used the symbol \( \emptyset \)
  \item Natural Number Set 
    \begin{itemize}
      \item \( \mathbb{N} = \{1, 2, 3, 4 \ldots \}\)
    \end{itemize}
  \item Integer Set 
    \begin{itemize}
      \item \( \mathbb{Z} = \{\ldots -3, -2, -1, 0, 1, 2, 3 \ldots \} \)
    \end{itemize}
  \item Rational Number Set 
    \begin{itemize}
      \item \( \mathbb{Q} = \{ \frac{m}{n}: n \in \mathbb{N}, z \in \mathbb{Z}\} \)
    \end{itemize}
  \item Complex Number Set 
    \begin{itemize}
      \item \( \mathbb{C} = \{a \pm i * b : b \in \mathbb{R}\} \)
    \end{itemize}

  \item Cardinality of a Set 
    \begin{itemize}
      \item If S is a set \( |S| \) is it cardinality 
      \item \( |S| = \) Number of elements in s if s is infinite if s has infinitivly many elements 
      \item \( |\emptyset | = 0 \)
    \end{itemize}
    Subsets and Proper Subsets 
    \begin{itemize}
      \item \( A \) is a subset of \( B \) if every element of \( A \) is a element of \( B \)
      \item There are also proper subsets and normal subsets, but people often don't mind the difference unless there is a explicit reason for needing one over the other \( \subset \)
    \end{itemize}
\end{itemize}

\subsection{Power Set}
If \( A \) is a set then the set of all subsets of \( A \) is called a powerset of \( A \) and denoted by \( \mathbb{P} \)
\section{Important Formulas/Theorems/Definitions}
\begin{tcolorbox}[colback=blue!5!white,colframe=blue!75!black,title=Key Formula/Theorem]
  In general \begin{equation}
    |\mathbb{P}(A)| = 2^{|A|}
  \end{equation}
\end{tcolorbox}

\section{Examples}
\begin{tcolorbox}
  D is a set of digits 
  \begin{itemize}
    \item $D = \{0, 1, 2, 3, 4, 5, 6, 7, 8, 9\}$ 
    \item $D = \{0, 1, 2, 3 \ldots 8, 9\}$
    \item \( D = \{ x : x\text{ is a digit} \} \)
    \item \( D = \{ x | x \text { is a digit}\} \)
  \end{itemize}
  Cardinality of a Set 
  \begin{itemize}
    \item \( |D| = 10 \)
    \item  \( s = \{ x: |x| =2 \} \)
      \begin{itemize}
        \item \( |S| = 2 \)
      \end{itemize}
  \end{itemize}
  Power Set
  \begin{itemize}
    \item \( A = \{ 1, 2, 3\} \)
  \item \( \mathbb{P}{A} = \{\emptyset , \{1\}, \{2\}, \{3\}, \{1,2\}, \{1,3\}, \{2,3\}, \{1,2,3\}\} \)
  \item \( |\mathbb{P} (A)| = 8\)
  \end{itemize}
\end{tcolorbox}

\section{Questions/Topics for Further Study}
\begin{itemize}
  \item What are the other possible applications of cardinality other than just the size of the set 
  \item What are the applications of power sets 
  \item 
\end{itemize}

\end{document}
